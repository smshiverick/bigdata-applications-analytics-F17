\documentclass[sigconf]{acmart}

\input{format/i523}

\begin{document}
\title{Using Machine Learning Classification of Opioid Addiction
for Big Data Health Analytics}

  \author{Sean M. Shiverick}
  \affiliation{%
  \institution{Indiana University Bloomington}
}
\email{smshiver@indiana.edu}

\renewcommand{\shortauthors}{S.M. Shiverick}

%%%%%%%%%%%%%%%%%%%%%%%%%%%%%%%%%%%%%%%%%%%%%%%%%%%%%%%%%%%%%%%%%%%%%%%%%%%%%%%%

\begin{abstract}
Classification of opioid addiction can help identify features related to 
drug abuse and overdose death. Machine learning procedures were used on data 
from a large National Survey of Drug Use and Health (NSDUH-2015) to classify 
individuals for illicit opioid use according to demographic characteristics 
and mental health attributes (i.e., depression). The classification model of 
opioid addiction could be extended for big data health analytics to include 
data over previous years of the survey, or extending the sample to the larger 
population of patients taking prescription opioid medication. Results from  
this project are intended to raise awareness of risk factors related to 
opioid addiction among patients and medication prescribers, and provide 
information that may help to decrease the risk of opioid overdose and death. 
Different classification methods are discussed.

\end{abstract}

\keywords{Health Analytics, Machine Learning, Classification, Opioid Addiction, 
Big Data, i523, HID335}

\maketitle

\url{url: https://github.com/bigdata-i523/hid335/tree/master/project}

%%%%%%%%%%%%%%%%%%%%%%%%%%%%%%%%%%%%%%%%%%%%%%%%%%%%%%%%%%%%%%%%%%%%%%%%%%%%%%%%
\section{Introduction}

The abuse of prescription opioid medication has become a major health crisis 
in the U.S. that has reached epidemic proportions \cite{volkow14}.In 2014, 
over 2 million Americans were dependent or abused prescription opioids such 
as oxycodone (i.e. ``Oxycontin'') or hydrocodone (e.g., ``Vicodan'') 
\cite{cdc17}. Overdose deaths from prescription opioids have quadrupled since 
1999, resulting in more than 180,000 deaths between 1999 to 2015 \cite{nida17}. 
Mobile health applications can help monitor medication adherence and provide 
an early warning system for potential abuse of prescription medication 
\cite{varshney13}. Medication abuse can consist of higher medication dosages 
or rapid escalation of a prescribed dosage. The general goal of this prediction 
model is to analyze patient data for sudden changes in medication consumption;
however, reliable information about medication consumption can be difficult 
to obtain from potentially addicted individuals based on self-reports. 
Individuals dependent or addicted to opioids may seek to obtain medications 
without a prescription, seek synthetic opioids such as fentanyl, or obtain 
illicit drugs such as heroin. The goal of this project was to use machine
learning classification procedures to help identify demographic information 
and features of individuals who are potentially susceptible to opioid 
dependency, abuse, and addiction.Classification of illicit opioid use
could help identify at risk for developing opioid dependency and addiction. 

%%%%%%%%%%%%%%%%%%%%%%%%%%%%%%%%%%%%%%%%%%%%%%%%%%%%%%%%%%%%%%%%%%%%%%%%%%%%%%%%
\subsection{Opioid Abuse, Dependency, and Addiction}

Drug addiction has many similar characteristics to other chronic medical 
illnesses, but there are unique challenges to the treatment of addiction
\cite{marsch12, swendson16}. In drug rehabilitation treatment programs, 
patients undergo intense detoxification that reduces their drug tolerance, but 
are then released back into the environments associated with their drug use, 
putting them at high risk for relapse and potential drug overdose 
\cite{johnson11}. According to a classical conditioning model of addiction, 
situational cues or events can elicit a motivational state underlying relapse 
to drug use. Addictive behavior can be also be reinstated after extinction of 
dependency by exposure to drug-related cues or stressors in the environment 
\cite{shaham03}. Categorizing or classifying cases of opioid abuse can 
contribute to efforts at addressing the opioid crisis in the following ways: 
(i) Identify factors related to opioid dependency, (ii) Inform consumers of 
opioid medication about relevant risk factors, and (iii) Increase knowledge 
of opioid abuse for more informed decisions by prescribers. 

%%%%%%%%%%%%%%%%%%%%%%%%%%%%%%%%%%%%%%%%%%%%%%%%%%%%%%%%%%%%%%%%%%%%%%%%%%%%%%%%
\subsection{Machine Learning Approaches} 

Machine learning is a set of procedures and automated processes for extracting 
knowledge from data. The two main branches of machine learning are' supervised 
learning and unsupervised learning. Supervised learning is used with problems 
that involve prediction about a specific target variable or outcome of interest. 
If a given dataset has no target outcome, unsupervised learning can be used to 
discover underlying structure in unlabeled data. The goal of this project is 
to classify opioid addiction and focuses on supervised learning. Supervised 
learning is used to predict a certain outcome from a given input, when examples 
of input/output pairs are available \cite{muller17}. A machine learning model 
is constructed from the training set of input-output pairs, to predict new test 
data not previously seen by the model. The two major approaches to supervised 
learning problems are regression and classification. When the target variable 
to be predicted is continuous, or there is continuity between the outcome (e.g., 
home values, or income), a regression model is used to test the set of features 
that predict the target variable. If the target is a class label, set of 
categorical or binary outcomes (e.g., `spam` or `ham`, `benign` or `malignant`), 
then classification is used to predict which class or category label that new 
instances will be assigned to.

\subsection{Classification Algorithms} 
Comparing the performance of different learning algorithms can be helpful for 
selecting the best model for a given problem \cite{raschka17}. One of the 
simplest classification algorithms is K-Nearest-Neighbors (KNN) which takes a 
set of data points and classifies a new data point based on the distance 
(Euclidean by default) to its nearest neighbor(s). The main parameter for KNN
is the number of neighbors, and typically k equal to 3 or 5 works well. KNN is
easy to understand and performs well with little adjustment, but it doe not 
perform well with datasets that have a large number of features (100 or more) 
or data that is sparse. For linear classification models, the decision boundary 
is a linear function of the input; a binary classifier separates two classes 
using along a line, plane, or hyperplane. Algorithms for linear classification 
differ in terms of (1) how they measure how well a particular combination of 
coefficients and intercept fit the training data, and (2) the type of 
regularization used \cite{muller17}. Two common linear classification models 
are logistic regression and support vector machines (SVMs). 


Decision Trees




%%%%%%%%%%%%%%%%%%%%%%%%%%%%%%%%%%%%%%%%%%%%%%%%%%%%%%%%%%%%%%%%%%%%%%%%%%%%%%%%
\subsection{Project Goals} 

The data for this project is the 2015 National Survey on Drug Use and Health 
(NSHUH-2015) completed by the U. S. Department of Health and Human Services 
(DHHS), Substance Abuse and Mental Health Services Administration (SAMHSA), 
Center for Behavioral Health Statistics and Quality \cite{samhsa16}. The 
NSDUH-2015 is a comprehensive survey that cover all aspects of substance use, 
misuse, abuse, including questions related to both prescription medications 
(opioids, tranquilizers, sedatives) and illicit drugs (e.g., heroin, cocaine, 
methamphetamine), drug dependency, addiction, and treatment, demographic 
measures of education and employment, physical health, depression, and mental 
health treatment. The goal is to identify the set of features that best 
classifies opioid addiction, defined by lifetime heroin. A machine learning 
classification model will be constructed to classify individuals by heroin 
use according to demographics attributes and mental health characteristics 
(e.g., adult depression). Understanding the features that contribute to 
opioid addiction may help to identify risk factors and increase awareness
of the potential of opioid abuse by patients and health care providers. 

%%%%%%%%%%%%%%%%%%%%%%%%%%%%%%%%%%%%%%%%%%%%%%%%%%%%%%%%%%%%%%%%%%%%%%%%%%%%%%%%
\section{Method}

\subsection{Data} 

Data from the 2015 NSHUH was downloaded from the Substance Abuse and Mental 
Health Data Archive (SAMHDA) as a tab separated Excel file \cite{samhsa16}. 
The dataset consists of 57,146 observations with 2,666 features representing 
individual-level responses from a survey of the U.S. population. Sampling was 
weighted across states by population size for a representative distribution 
selected from 6,000 area segments varying in size according to state. The 
sample design used five state sample size groups drawing more heavily from 
the eight states with the largest population (e.g., CA, FL, IL, MI, NY, OH, 
PA, TX) which together account for 48 percent of total U.S. population aged 
12 or older. Given that NSDUH collects highly-sensitive information from 
individuals, all identifying information was collapsed (e.g., age categories)
and state identifiers were removed from the public use file to ensure 
confidentiality. The NSDUH public-use files do not include geographic location, 
or demographic variables related to ethnicity or immigration status. The 
weighted survey screening response rate was 81.94 percent and the weighted
interview response rate was 71.2 percent. The NSDUH-2015 dataset was 
downloaded from the SAMH Data Archive as tab-separated Excel files.


%%%%%%%%%%%%%%%%%%%%%%%%%%%%%%%%%%%%%%%%%%%%%%%%%%%%%%%%%%%%%%%%%%%%%%%%%%%%%%%%
\subsection{Data Cleaning and Preparation }

\subsubsection{Data Cleaning}
This analysis was completed in a python interactive notebook \cite{data17} 
based on instructions and examples from \emph{Python for Data Analysis}
\cite{mckinney17}. The NSDUH-2015 was loaded into python using Pandas and 
saved as a dataframe object. The dataset was subset by columns to include: 
demographic characteristics (e.g., age category, sex, marital status, education, 
employment status, and category of metropolitan area), measures of physical 
health (e.g., overall health, STDs, Hepatitis, HIV, Cancer, hospitalization), 
mental health (e.g., Adult Depression, Emotional Distress, Suicidal Thoughts, 
Plans), Suicide Attempts, Pain Reliever Medication Use, Misuse, and Abuse
(over the past year, past month), Prescription Opioid Medications Taken in
the Past year (e.g., Hydrocodone, Oxycodone, Tramadol, Morphine, Fentanyl,
Oxymorphone, Demerol, Hydromorphone), Heroin Use, Abuse, and Dependency
(over the past year, past month), Tranquilizer Use, Sedative Use, Cocaine 
Use, Amphetamine and Methamphetamine Use, Hallucinogen Use, Drug Treatment
(e.g., Inpatient, Outpatient, Hospital, Mental Health Clinic, ER, Drug
Treatment Status), and Mental Health Treatment History. A codebook was 
created to provide a complete list of variables included with summaries 
of response categories \cite{codebook17}. Several steps to taken to detect 
and remove inconsistencies in the data \cite{rahm00}: 
\begin{enumerate}
  \item Missing values (i.e., `NaN`) were removed 
  \item Blanks, non-responses, or legitimate skips were recoded to zero 
  (e.g.,`99`, `991`, `993`) 
  \item Dichotomous responses (e.g., ``Yes=1``/``No=2``) were responses to 
  ``No=0``
  \item Categorical variables so the response was consistent with degree 
  (e.g., ``1=low``, ``2=med``, ``3=high``)
   \item Selected variables were renamed for better description (e.g., 
   Adult Major Depressive Episode Lifetime changed from `AMDELT` to `DEPMELT`)
\end{enumerate}

Because the majority of features were represented as dichotomous ``Yes/No`` 
variables, related features were summed to create aggregated variables. For 
example, overall health, STD, Hepatitis, HIV, Cancer, and hospitalization were 
aggregated to create a single health measure. The health measure was recoded
so that higher scores indicated better health. Questions related to depression, 
emotional distress, and suicidal thoughts were summed to create a single 
variable for mental health (`MENTHLTH`) with scores ranging from 0 to 9. 
Responses to pain reliever medication use, misuse, abuse, or dependency, 
were aggregated to create a single variable of pain reliever misuse or abuse
(`PRLMISAB`). All prescription painkiller medications used in the past year
were summed. Similarly, all related responses were summed to create single 
variables for Tranquilizers, Sedatives, Cocaine, Amphetamines, Hallucinogens, 
Drug Treatment, and Mental Health Treatment. The target outcome of interest for 
classification, lifetime heroin use (i.e., ``Have you ever used heroin before, 
at any point in time?``) is a dichotomous variables. The demographic 
characteristics and aggregated variables were subset and saved to a new data 
frame consisting of 22 features and 57,146 observations, which was exported 
to CSV file, titled as `project-data.csv`. 


%%%%%%%%%%%%%%%%%%%%%%%%%%%%%%%%%%%%%%%%%%%%%%%%%%%%%%%%%%%%%%%%%%%%%%%%%%%%%%%%
\section{Results}

\subsection{Exploratory Analysis and Descriptives}




\cite{mckinney17}



%%%%%%%%%%%%%%%%%%%%%%%%%%%%%%%%%%%%%%%%%%%%%%%%%%%%%%%%%%%%%%%%%%%%%%%%%%%%%%%%
\section{Classification Algorithms}

\cite{muller17}


\cite{raschka17}




\cite{vanderplas17}


%%%%%%%%%%%%%%%%%%%%%%%%%%%%%%%%%%%%%%%%%%%%%%%%%%%%%%%%%%%%%%%%%%%%%%%%%%%%%%%%
\section{Discussion}

\subsection{Extension to Big Data}

This classification method could be adopted for big data health analytics by
expanding the suvery to the popluation of patients with prescriptions for 
opioid pain medication and may help to identify individuals at risk for
potentially developing opioid dependency of addiction. 


Additional data from the NSDUH was downloaded from 2012 to 2014

hen trying to integrate data from multiple sources, (4) inconsistencies in coding of attributes or features can be a major source of problems. For example, time stamped data with different date formats (e.g., DD/MM/YY; YY/MM/DD, or MM/DD/YY) would have to be reformatted before being combined. Similarly, different coding for sex (e.g., ‘M’/’F’, ‘0’/’1’, ‘1’,’2’) have to be reconciled when combining multiple data sources. Cleaning big data is a more involved process, with Extraction, Integration, and Aggregation of data from operationalized sources into a data warehouse. In general, there are several phases involved in big data cleaning approaches, including the following:



%%%%%%%%%%%%%%%%%%%%%%%%%%%%%%%%%%%%%%%%%%%%%%%%%%%%%%%%%%%%%%%%%%%%%%%%%%%%%%%%

\section{figures}

In Figure \ref{f:fly} we show a fly. Please note that because we use
just columwidth that the size of the figure will change to the
columnwidth of the paper once we change the layout to final. 

\begin{figure}[!ht]
  \centering\includegraphics[width=\columnwidth]{images/fly.pdf}
  \caption{Example caption}\label{f:fly}
\end{figure}


%%%%%%%%%%%%%%%%%%%%%%%%%%%%%%%%%%%%%%%%%%%%%%%%%%%%%%%%%%%%%%%%%%%%%%%%%%%%%%%%
\section{Tables}

%Table\ref{t:mytable}. 

\begin{table}[htb]
\centering
\caption{My caption}
\label{t:mytabble}
\begin{tabular}{lll}
1 & 2 & 3 \\
\hline
4 & 5 & 6 \\
7 & 8 & 9
\end{tabular}
\end{table}

\section{Long example}

If you like to see a more elaborate example, please look at
report-long.tex. 

%%%%%%%%%%%%%%%%%%%%%%%%%%%%%%%%%%%%%%%%%%%%%%%%%%%%%%%%%%%%%%%%%%%%%%%%%%%%%%%%
\section{Conclusion}

Machine learning classification of opioid abuse can contribute to efforts to address prescription opioid addiction, overdoses, in the following ways: 
\begin{enumerate}
\item Identify factors related to opioid dependency
\item Inform consumers of opioid medication as to risk factors 
\item Increase knowledge of opioid abuse for more informed prescriptions. 
\end{enumerate}

%%%%%%%%%%%%%%%%%%%%%%%%%%%%%%%%%%%%%%%%%%%%%%%%%%%%%%%%%%%%%%%%%%%%%%%%%%%%%%%%
\begin{acks}

  The author would like to thank Dr. Gregor von Laszewski, 
  the Assistant Instructors, Juliette Zurick, Miao Zheng,
  and others who helped to improve this project and report.

\end{acks}

\bibliographystyle{ACM-Reference-Format}
\bibliography{report} 


%%%%%%%%%%%%%%%%%%%%%%%%%%%%%%%%%%%%%%%%%%%%%%%%%%%%%%%%%%%%%%%%%%%%%%%%%%%%%%%%
\appendix

\section{Code References}
References used in data cleaning and preparation, exploratory analysis, 
visualization, classification model construction and evaluation are documented
here as well as included in comments in the jupyter notebooks 
\cite{muller17,mckinney17,vanderplas17}

Appendix of common issues: 

%\begin{enumerate}

%\item Do not to use the underscore in bibtex labels
%\item Address each of the items in the issues.tex file  and verify you have done them. 
%item Please do this only at the end once you have finished writing the paper. 
%\item Change `TODO` with `DONE`. 

%\end{enumerate}

%\section{Issues}

\DONE{Example of done item: Once you fix an item, change TODO to DONE}

\subsection{Assignment Submission Issues}

    \DONE{Do not make changes to your paper during grading, when your repository should be frozen.}

\subsection{Uncaught Bibliography Errors}

    \DONE{Missing bibliography file generated by JabRef}
    \DONE{Bibtex labels cannot have any spaces, \_ or \& in it}
    \DONE{Citations in text showing as [?]: this means either your report.bib is not up-to-date or there is a spelling error in the label of the item you want to cite, either in report.bib or in report.tex}

\subsection{Formatting}

    \DONE{Incorrect number of keywords or HID and i523 not included in the keywords}
    \DONE{Other formatting issues}

\subsection{Writing Errors}

    \DONE{Errors in title, e.g. capitalization}
    \DONE{Spelling errors}
    \DONE{Are you using {\em a} and {\em the} properly?}
    \DONE{Do not use phrases such as {\em shown in the Figure below}. Instead, use {\em as shown in Figure 3}, when referring to the 3rd figure}
    \DONE{Do not use the word {\em I} instead use {\em we} even if you are the sole author}
    \DONE{Do not use the phrase {\em In this paper/report we show} instead use {\em We show}. It is not important if this is a paper or a report and does not need to be mentioned}
    \DONE{If you want to say {\em and} do not use {\em \&} but use the word {\em and}}
    \DONE{Use a space after . , : }
    \DONE{When using a section command, the section title is not written in all-caps as format does this for you}\begin{verbatim}\section{Introduction} and NOT \section{INTRODUCTION} \end{verbatim}

\subsection{Citation Issues and Plagiarism}

    \DONE{It is your responsibility to make sure no plagiarism occurs. The instructions and resources were given in the class}
    \DONE{Claims made without citations provided}
    \DONE{Need to paraphrase long quotations (whole sentences or longer)}
    \DONE{Need to quote directly cited material}

\subsection{Character Errors}

    \DONE{Erroneous use of quotation marks, i.e. use ``quotes'' , instead of " "}
    \DONE{To emphasize a word, use {\em emphasize} and not ``quote''}
    \DONE{When using the characters \& \# \% \_  put a backslash before them so that they show up correctly}
    \DONE{Pasting and copying from the Web often results in non-ASCII characters to be used in your text, please remove them and replace accordingly. This is the case for quotes, dashes and all the other special characters.}
    \DONE{If you see a figure and not a figure in text you copied from a text that has the fi combined as a single character}

\subsection{Structural Issues}

    \DONE{Acknowledgement section missing}
    \DONE{Incorrect README file}
    \DONE{In case of a class and if you do a multi-author paper, you need to add an appendix describing who did what in the paper}
    \DONE{The paper has less than 2 pages of text, i.e. excluding images, tables and figures}
    \DONE{The paper has more than 6 pages of text, i.e. excluding images, tables and figures}
    \DONE{Do not artificially inflate your paper if you are below the page limit}

\subsection{Details about the Figures and Tables}

    \DONE{Capitalization errors in referring to captions, e.g. Figure 1, Table 2}
    \DONE{Do use {\em label} and {\em ref} to automatically create figure numbers}
    \DONE{Wrong placement of figure caption. They should be on the bottom of the figure}
    \DONE{Wrong placement of table caption. They should be on the top of the table}
    \DONE{Images submitted incorrectly. They should be in native format, e.g. .graffle, .pptx, .png, .jpg}
    \DONE{Do not submit eps images. Instead, convert them to PDF}

    \DONE{The image files must be in a single directory named "images"}
    \DONE{In case there is a powerpoint in the submission, the image must be exported as PDF}
    \DONE{Make the figures large enough so we can read the details. If needed make the figure over two columns}
    \DONE{Do not worry about the figure placement if they are at a different location than you think. Figures are allowed to float. For this class, you should place all figures at the end of the report.}
    \DONE{In case you copied a figure from another paper you need to ask for copyright permission. In case of a class paper, you must include a reference to the original in the caption}
    \DONE{Remove any figure that is not referred to explicitly in the text (As shown in Figure ..)}
    \DONE{Do not use textwidth as a parameter for includegraphics}
    \DONE{Figures should be reasonably sized and often you just need to
    \DONE{includegraphics[width=\columnwidth]{images/myimage.pdf}}
=======
    \TODO{Do not make changes to your paper during grading, when your repository should be frozen.}

\subsection{Uncaught Bibliography Errors}

    \TODO{Missing bibliography file generated by JabRef}
    \TODO{Bibtex labels cannot have any spaces, \_ or \& in it}
    \TODO{Citations in text showing as [?]: this means either your report.bib is not up-to-date or there is a spelling error in the label of the item you want to cite, either in report.bib or in report.tex}

\subsection{Formatting}

    \TODO{Incorrect number of keywords or HID and i523 not included in the keywords}
    \TODO{Other formatting issues}

\subsection{Writing Errors}

    \TODO{Errors in title, e.g. capitalization}
    \TODO{Spelling errors}
    \TODO{Are you using {\em a} and {\em the} properly?}
    \TODO{Do not use phrases such as {\em shown in the Figure below}. Instead, use {\em as shown in Figure 3}, when referring to the 3rd figure}
    \TODO{Do not use the word {\em I} instead use {\em we} even if you are the sole author}
    \TODO{Do not use the phrase {\em In this paper/report we show} instead use {\em We show}. It is not important if this is a paper or a report and does not need to be mentioned}
    \TODO{If you want to say {\em and} do not use {\em \&} but use the word {\em and}}
    \TODO{Use a space after . , : }
    \TODO{When using a section command, the section title is not written in all-caps as format does this for you}\begin{verbatim}\section{Introduction} and NOT \section{INTRODUCTION} \end{verbatim}

\subsection{Citation Issues and Plagiarism}

    \TODO{It is your responsibility to make sure no plagiarism occurs. The instructions and resources were given in the class}
    \TODO{Claims made without citations provided}
    \TODO{Need to paraphrase long quotations (whole sentences or longer)}
    \TODO{Need to quote directly cited material}

\subsection{Character Errors}

    \TODO{Erroneous use of quotation marks, i.e. use ``quotes'' , instead of " "}
    \TODO{To emphasize a word, use {\em emphasize} and not ``quote''}
    \TODO{When using the characters \& \# \% \_  put a backslash before them so that they show up correctly}
    \TODO{Pasting and copying from the Web often results in non-ASCII characters to be used in your text, please remove them and replace accordingly. This is the case for quotes, dashes and all the other special characters.}
    \TODO{If you see a figure and not a figure in text you copied from a text that has the fi combined as a single character}

\subsection{Structural Issues}

    \TODO{Acknowledgement section missing}
    \TODO{Incorrect README file}
    \TODO{In case of a class and if you do a multi-author paper, you need to add an appendix describing who did what in the paper}
    \TODO{The paper has less than 2 pages of text, i.e. excluding images, tables and figures}
    \TODO{The paper has more than 6 pages of text, i.e. excluding images, tables and figures}
    \TODO{Do not artificially inflate your paper if you are below the page limit}

\subsection{Details about the Figures and Tables}

    \TODO{Capitalization errors in referring to captions, e.g. Figure 1, Table 2}
    \TODO{Do use {\em label} and {\em ref} to automatically create figure numbers}
    \TODO{Wrong placement of figure caption. They should be on the bottom of the figure}
    \TODO{Wrong placement of table caption. They should be on the top of the table}
    \TODO{Images submitted incorrectly. They should be in native format, e.g. .graffle, .pptx, .png, .jpg}
    \TODO{Do not submit eps images. Instead, convert them to PDF}

    \TODO{The image files must be in a single directory named "images"}
    \TODO{In case there is a powerpoint in the submission, the image must be exported as PDF}
    \TODO{Make the figures large enough so we can read the details. If needed make the figure over two columns}
    \TODO{Do not worry about the figure placement if they are at a different location than you think. Figures are allowed to float. For this class, you should place all figures at the end of the report.}
    \TODO{In case you copied a figure from another paper you need to ask for copyright permission. In case of a class paper, you must include a reference to the original in the caption}
    \TODO{Remove any figure that is not referred to explicitly in the text (As shown in Figure ..)}
    \TODO{Do not use textwidth as a parameter for includegraphics}
    \TODO{Figures should be reasonably sized and often you just need to
    \TODO{Includegraphics[width=\columnwidth]{images/myimage.pdf}}

re


\end{document}
