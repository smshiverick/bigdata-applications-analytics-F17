\documentclass[sigconf]{acmart}

\input{format/i523}

\begin{document}

\title{Machine Learning Classification of Opioid Use and Addiction
for Big Data Health Analytics}

  \author{Sean M. Shiverick}
  \affiliation{%
  \institution{Indiana University Bloomington}
}
\email{smshiver@indiana.edu}

%%%%%%%%%%%%%%%%%%%%%%%%%%%%%%%%%%%%%%%%%%%%%%%%%%%%%%%%%%%%%%%%%%%%%%%%%%%%%%%%

\begin{abstract}
Classification of opioid abuse and dependency can help to identify features
relevant to drug addiction and overdose. This project uses machine learning
procedures to classify individuals according to illicit opioid use, and
identify characteristics of individuals susceptible to opioid addiction.
Developing a classification model for identifying factors related to opioid 
addiction could be extended over previous years, to the population of
patients who are prescribed opioid medication, to help reduce rates of 
opioid addiction, and curtain the epidemic of opioid medication abuse.  
Differences between the classification methods is discussed.

\end{abstract}

\keywords{Health Analytics, Machine Learning, Classification, Opioid Addiction, 
Big Data, i523, HID335}

\maketitle

%%%%%%%%%%%%%%%%%%%%%%%%%%%%%%%%%%%%%%%%%%%%%%%%%%%%%%%%%%%%%%%%%%%%%%%%%%%%%%%%
\section{Introduction}

The abuse of prescription opioid medication has become a major health crisis 
in the U.S. that has reached epidemic proportion \cite{volkow14}. Over 2 
million Americans were dependent or abused prescription opioids such as 
oxycodone (i.e. ``Oxycontin'') or hydrocodone (e.g., ``Vicodan'') in 2014 
\cite{cdc17}. Overdose deaths from prescription opioids has quadrupled since 
1999, resulting in more than 180,000 deaths between 1999 to 2015 \cite{nida17}. 
Mobile health applications can help monitor medication adherence and provide 
an early warning system for potential abuse of prescription medication 
\cite{varshney13}. Medication abuse can consist of higher medication dosages 
or rapid escalation of a prescribed dosage. The general goal of this prediction 
model is to analyze patient data for sudden changes in medication consumption;
however, reliable information about medication consumption can be difficult 
to obtain from potentially addicted individuals based on self-reports. 
Individuals dependent or addicted to opioids may seek to obtain medications 
without a prescription, seek synthetic opioids such as fentanyl, or obtain 
illicit drugs such as heroin. The goal of this project was to use machine
learning classification procedures to help identify demographic information 
and features of individuals who are potentially susceptible to opioid 
dependency, abuse, and addiction.Classification of illicit opioid use
could help identify at risk for developing opioid dependency and addiction. 

\subsection{Drug Addiction and Treatment}









Addiction is a complex phenomenon, with multiple and interacting factors. 


Drug addiction chronic health conditions 
Drug addiction has many similar characteristics to other chronic medical 
illnesses; however, there are unique challenges to the treatment of addiction
illnesses \cite{boyer10}. 

For example, drug addicted patients undergo intense detoxification 
in rehabilitation treatment programs, which reduces their drug tolerance, and 
then are released back into the same environment associated with their drug use, 
putting them at greater risk for relapse and potential drug overdose. 


Efforts to address prescription opioid abuse, overdoses, and deaths have 
focused on four major risk areas: (i) Increasing knowledge of opioid abuse 
and improving decisions of medication prescribers, (ii) Reducing inappropriate 
access to opioids, (iii) Increasing effective overdose treatment, (iv) 
Providing substance-abuse treatment to persons addicted to opioids. 


The lack 
of continuity in the treatment of addiction disorders leaves persons in recovery 
at high risk of relapse for substance use and abuse. 

have been proposed. According to the classical conditioning model, situational 
cues or events can elicit a motivational state underlying relapse to drug use. 

A slightly more complex model suggests that addictive behavior can be reinstated 
after extinction of dependency by exposure to drugs, drug-related cues, or 
environmental stressors \cite{shaham03}. 



\cite{cdc17}.


\cite{johnson11}
\cite{marsch12}
\cite{nida17}


\subsection{Medication Abuse and Addiction} 






%%%%%%%%%%%%%%%%%%%%%%%%%%%%%%%%%%%%%%%%%%%%%%%%%%%%%%%%%%%%%%%%%%%%%%%%%%%%%%%%
\section{Project Data: NSDUH-2015} 

The data for this project were obtained from the National Survey on Drug Use 
and Health 







%%%%%%%%%%%%%%%%%%%%%%%%%%%%%%%%%%%%%%%%%%%%%%%%%%%%%%%%%%%%%%%%%%%%%%%%%%%%%%%%
\section{Data Cleaning and Preparation}\cite{rahm00}




%%%%%%%%%%%%%%%%%%%%%%%%%%%%%%%%%%%%%%%%%%%%%%%%%%%%%%%%%%%%%%%%%%%%%%%%%%%%%%%%


%%%%%%%%%%%%%%%%%%%%%%%%%%%%%%%%%%%%%%%%%%%%%%%%%%%%%%%%%%%%%%%%%%%%%%%%%%%%%%%%





%%%%%%%%%%%%%%%%%%%%%%%%%%%%%%%%%%%%%%%%%%%%%%%%%%%%%%%%%%%%%%%%%%%%%%%%%%%%%%%%
\section{figures}

In Figure \ref{f:fly} 


we show a fly. Please note that because we use
just columwidth that the size of the figure will change to the
columnwidth of the paper once we change the layout to final. CHnaging
the layout to final should not be done by you. All figures will be
listed at the end.

\begin{figure}[!ht]
  \centering\includegraphics[width=\columnwidth]{images/fly.pdf}
  \caption{Example caption}\label{f:fly}
\end{figure}

When copying the example, please do not check in the images from the
examples into your images directory as you will not need them for your
paper. Instead use images that you like to include. If you do not have
any images, do not dreate the images folder.

%%%%%%%%%%%%%%%%%%%%%%%%%%%%%%%%%%%%%%%%%%%%%%%%%%%%%%%%%%%%%%%%%%%%%%%%%%%%%%%%
\section{Tables}

In case you need to create tables, you can do this with online tools
(if you do not mind sharing your data) such as
\url{https://www.tablesgenerator.com/} or other such tools (please
google for them). They even allow you to manage tables as CSV.

or generate them by hand while using the provided template in Table\ref{t:mytable}. Not ethat
the caption is before the tabular environment.

\begin{table}[htb]
\centering
\caption{My caption}
\label{t:mytabble}
\begin{tabular}{lll}
1 & 2 & 3 \\
\hline
4 & 5 & 6 \\
7 & 8 & 9
\end{tabular}
\end{table}

\section{Long example}

If you like to see a more elaborate example, please look at
report-long.tex. 

\section{Conclusion}

Put here an conclusion. Conlcusions and abstracts must not have any
citations in the section.


\begin{acks}

  The author would like to thank Dr. Gregor von Laszewski, 
    the Assistant Instructors, Juliette Zurick, Jiang Miao,
    and others who helped to improve this report. 

\end{acks}

\bibliographystyle{ACM-Reference-Format}
\bibliography{report} 


%%%%%%%%%%%%%%%%%%%%%%%%%%%%%%%%%%%%%%%%%%%%%%%%%%%%%%%%%%%%%%%%%%%%%%%%%%%%%%%%%%

\appendix

\section{Code References}

Appendix of common issues: 

\begin{enumerate}

\item Cloud - chameleon, jetstream, or kilo (futuresystems)
\item MongoDB Version - 34 for version 3.4, 32 for version 3.2
\item Config Server Replication Size - a number
\item Mongos Router Instances - a number
\item Shard Count - a number
\item Shard Replication Size - a number

\end{enumerate}


Do not to use the underscore in bibtex labels.  
Address each of the items in the issues.tex file 
and verify that you have done them. 
Please do this only at the end once you have 
finished writing the paper. 
Change `TODO` with `DONE`. 





%\section{Issues}

\DONE{Example of done item: Once you fix an item, change TODO to DONE}

\subsection{Assignment Submission Issues}

    \DONE{Do not make changes to your paper during grading, when your repository should be frozen.}

\subsection{Uncaught Bibliography Errors}

    \DONE{Missing bibliography file generated by JabRef}
    \DONE{Bibtex labels cannot have any spaces, \_ or \& in it}
    \DONE{Citations in text showing as [?]: this means either your report.bib is not up-to-date or there is a spelling error in the label of the item you want to cite, either in report.bib or in report.tex}

\subsection{Formatting}

    \DONE{Incorrect number of keywords or HID and i523 not included in the keywords}
    \DONE{Other formatting issues}

\subsection{Writing Errors}

    \DONE{Errors in title, e.g. capitalization}
    \DONE{Spelling errors}
    \DONE{Are you using {\em a} and {\em the} properly?}
    \DONE{Do not use phrases such as {\em shown in the Figure below}. Instead, use {\em as shown in Figure 3}, when referring to the 3rd figure}
    \DONE{Do not use the word {\em I} instead use {\em we} even if you are the sole author}
    \DONE{Do not use the phrase {\em In this paper/report we show} instead use {\em We show}. It is not important if this is a paper or a report and does not need to be mentioned}
    \DONE{If you want to say {\em and} do not use {\em \&} but use the word {\em and}}
    \DONE{Use a space after . , : }
    \DONE{When using a section command, the section title is not written in all-caps as format does this for you}\begin{verbatim}\section{Introduction} and NOT \section{INTRODUCTION} \end{verbatim}

\subsection{Citation Issues and Plagiarism}

    \DONE{It is your responsibility to make sure no plagiarism occurs. The instructions and resources were given in the class}
    \DONE{Claims made without citations provided}
    \DONE{Need to paraphrase long quotations (whole sentences or longer)}
    \DONE{Need to quote directly cited material}

\subsection{Character Errors}

    \DONE{Erroneous use of quotation marks, i.e. use ``quotes'' , instead of " "}
    \DONE{To emphasize a word, use {\em emphasize} and not ``quote''}
    \DONE{When using the characters \& \# \% \_  put a backslash before them so that they show up correctly}
    \DONE{Pasting and copying from the Web often results in non-ASCII characters to be used in your text, please remove them and replace accordingly. This is the case for quotes, dashes and all the other special characters.}
    \DONE{If you see a figure and not a figure in text you copied from a text that has the fi combined as a single character}

\subsection{Structural Issues}

    \DONE{Acknowledgement section missing}
    \DONE{Incorrect README file}
    \DONE{In case of a class and if you do a multi-author paper, you need to add an appendix describing who did what in the paper}
    \DONE{The paper has less than 2 pages of text, i.e. excluding images, tables and figures}
    \DONE{The paper has more than 6 pages of text, i.e. excluding images, tables and figures}
    \DONE{Do not artificially inflate your paper if you are below the page limit}

\subsection{Details about the Figures and Tables}

    \DONE{Capitalization errors in referring to captions, e.g. Figure 1, Table 2}
    \DONE{Do use {\em label} and {\em ref} to automatically create figure numbers}
    \DONE{Wrong placement of figure caption. They should be on the bottom of the figure}
    \DONE{Wrong placement of table caption. They should be on the top of the table}
    \DONE{Images submitted incorrectly. They should be in native format, e.g. .graffle, .pptx, .png, .jpg}
    \DONE{Do not submit eps images. Instead, convert them to PDF}

    \DONE{The image files must be in a single directory named "images"}
    \DONE{In case there is a powerpoint in the submission, the image must be exported as PDF}
    \DONE{Make the figures large enough so we can read the details. If needed make the figure over two columns}
    \DONE{Do not worry about the figure placement if they are at a different location than you think. Figures are allowed to float. For this class, you should place all figures at the end of the report.}
    \DONE{In case you copied a figure from another paper you need to ask for copyright permission. In case of a class paper, you must include a reference to the original in the caption}
    \DONE{Remove any figure that is not referred to explicitly in the text (As shown in Figure ..)}
    \DONE{Do not use textwidth as a parameter for includegraphics}
    \DONE{Figures should be reasonably sized and often you just need to
    \DONE{includegraphics[width=\columnwidth]{images/myimage.pdf}}
=======
    \TODO{Do not make changes to your paper during grading, when your repository should be frozen.}

\subsection{Uncaught Bibliography Errors}

    \TODO{Missing bibliography file generated by JabRef}
    \TODO{Bibtex labels cannot have any spaces, \_ or \& in it}
    \TODO{Citations in text showing as [?]: this means either your report.bib is not up-to-date or there is a spelling error in the label of the item you want to cite, either in report.bib or in report.tex}

\subsection{Formatting}

    \TODO{Incorrect number of keywords or HID and i523 not included in the keywords}
    \TODO{Other formatting issues}

\subsection{Writing Errors}

    \TODO{Errors in title, e.g. capitalization}
    \TODO{Spelling errors}
    \TODO{Are you using {\em a} and {\em the} properly?}
    \TODO{Do not use phrases such as {\em shown in the Figure below}. Instead, use {\em as shown in Figure 3}, when referring to the 3rd figure}
    \TODO{Do not use the word {\em I} instead use {\em we} even if you are the sole author}
    \TODO{Do not use the phrase {\em In this paper/report we show} instead use {\em We show}. It is not important if this is a paper or a report and does not need to be mentioned}
    \TODO{If you want to say {\em and} do not use {\em \&} but use the word {\em and}}
    \TODO{Use a space after . , : }
    \TODO{When using a section command, the section title is not written in all-caps as format does this for you}\begin{verbatim}\section{Introduction} and NOT \section{INTRODUCTION} \end{verbatim}

\subsection{Citation Issues and Plagiarism}

    \TODO{It is your responsibility to make sure no plagiarism occurs. The instructions and resources were given in the class}
    \TODO{Claims made without citations provided}
    \TODO{Need to paraphrase long quotations (whole sentences or longer)}
    \TODO{Need to quote directly cited material}

\subsection{Character Errors}

    \TODO{Erroneous use of quotation marks, i.e. use ``quotes'' , instead of " "}
    \TODO{To emphasize a word, use {\em emphasize} and not ``quote''}
    \TODO{When using the characters \& \# \% \_  put a backslash before them so that they show up correctly}
    \TODO{Pasting and copying from the Web often results in non-ASCII characters to be used in your text, please remove them and replace accordingly. This is the case for quotes, dashes and all the other special characters.}
    \TODO{If you see a figure and not a figure in text you copied from a text that has the fi combined as a single character}

\subsection{Structural Issues}

    \TODO{Acknowledgement section missing}
    \TODO{Incorrect README file}
    \TODO{In case of a class and if you do a multi-author paper, you need to add an appendix describing who did what in the paper}
    \TODO{The paper has less than 2 pages of text, i.e. excluding images, tables and figures}
    \TODO{The paper has more than 6 pages of text, i.e. excluding images, tables and figures}
    \TODO{Do not artificially inflate your paper if you are below the page limit}

\subsection{Details about the Figures and Tables}

    \TODO{Capitalization errors in referring to captions, e.g. Figure 1, Table 2}
    \TODO{Do use {\em label} and {\em ref} to automatically create figure numbers}
    \TODO{Wrong placement of figure caption. They should be on the bottom of the figure}
    \TODO{Wrong placement of table caption. They should be on the top of the table}
    \TODO{Images submitted incorrectly. They should be in native format, e.g. .graffle, .pptx, .png, .jpg}
    \TODO{Do not submit eps images. Instead, convert them to PDF}

    \TODO{The image files must be in a single directory named "images"}
    \TODO{In case there is a powerpoint in the submission, the image must be exported as PDF}
    \TODO{Make the figures large enough so we can read the details. If needed make the figure over two columns}
    \TODO{Do not worry about the figure placement if they are at a different location than you think. Figures are allowed to float. For this class, you should place all figures at the end of the report.}
    \TODO{In case you copied a figure from another paper you need to ask for copyright permission. In case of a class paper, you must include a reference to the original in the caption}
    \TODO{Remove any figure that is not referred to explicitly in the text (As shown in Figure ..)}
    \TODO{Do not use textwidth as a parameter for includegraphics}
    \TODO{Figures should be reasonably sized and often you just need to
    \TODO{Includegraphics[width=\columnwidth]{images/myimage.pdf}}

re


\end{document}
