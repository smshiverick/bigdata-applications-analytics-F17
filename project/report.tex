\documentclass[sigconf]{acmart}

\input{format/i523}

\begin{document}
\title{Machine Learning Classification of Opioid Use and Addiction
for Big Data Health Analytics}

  \author{Sean M. Shiverick}
  \affiliation{%
  \institution{Indiana University Bloomington}
}
\email{smshiver@indiana.edu}

\renewcommand{\shortauthors}{S.M. Shiverick}

%%%%%%%%%%%%%%%%%%%%%%%%%%%%%%%%%%%%%%%%%%%%%%%%%%%%%%%%%%%%%%%%%%%%%%%%%%%%%%%%

\begin{abstract}
Classification of opioid abuse and dependency can help to identify features
relevant to drug addiction and overdose. This project uses machine learning
procedures to classify individuals according to illicit opioid use, and
identify characteristics of individuals susceptible to opioid addiction.
Developing a classification model for identifying factors related to opioid 
addiction could be extended over previous years, to the population of
patients who are prescribed opioid medication, to help reduce rates of 
opioid addiction, and curtain the epidemic of opioid medication abuse.  
Differences between the classification methods is discussed.

\end{abstract}

\keywords{Health Analytics, Machine Learning, Classification, Opioid Addiction, 
Big Data, i523, HID335}

\maketitle

\url{url: https://github.com/bigdata-i523/hid335/tree/master/project}

%%%%%%%%%%%%%%%%%%%%%%%%%%%%%%%%%%%%%%%%%%%%%%%%%%%%%%%%%%%%%%%%%%%%%%%%%%%%%%%%
\section{Introduction}

The abuse of prescription opioid medication has become a major health crisis 
in the U.S. that has reached epidemic proportions \cite{volkow14}.In 2014, 
over 2 million Americans were dependent or abused prescription opioids such 
as oxycodone (i.e. ``Oxycontin'') or hydrocodone (e.g., ``Vicodan'') 
\cite{cdc17}. Overdose deaths from prescription opioids have quadrupled since 
1999, resulting in more than 180,000 deaths between 1999 to 2015 \cite{nida17}. 
Mobile health applications can help monitor medication adherence and provide 
an early warning system for potential abuse of prescription medication 
\cite{varshney13}. Medication abuse can consist of higher medication dosages 
or rapid escalation of a prescribed dosage. The general goal of this prediction 
model is to analyze patient data for sudden changes in medication consumption;
however, reliable information about medication consumption can be difficult 
to obtain from potentially addicted individuals based on self-reports. 
Individuals dependent or addicted to opioids may seek to obtain medications 
without a prescription, seek synthetic opioids such as fentanyl, or obtain 
illicit drugs such as heroin. The goal of this project was to use machine
learning classification procedures to help identify demographic information 
and features of individuals who are potentially susceptible to opioid 
dependency, abuse, and addiction.Classification of illicit opioid use
could help identify at risk for developing opioid dependency and addiction. 

%%%%%%%%%%%%%%%%%%%%%%%%%%%%%%%%%%%%%%%%%%%%%%%%%%%%%%%%%%%%%%%%%%%%%%%%%%%%%%%%
\subsection{Opioid Abuse, Dependency, and Addiction}

Drug addiction has many similar characteristics to other chronic medical 
illnesses, but there are unique challenges to the treatment of addiction
\cite{marsch12, swendson16}. In drug rehabilitation treatment programs, 
patients undergo intense detoxification that reduces their drug tolerance, but 
are then released back into the environments associated with their drug use, 
putting them at high risk for relapse and potential drug overdose 
\cite{johnson11}. According to a classical conditioning model of addiction, 
situational cues or events can elicit a motivational state underlying relapse 
to drug use. Addictive behavior can be also be reinstated after extinction of 
dependency by exposure to drug-related cues or stressors in the environment 
\cite{shaham03}. Categorizing or classifying cases of opioid abuse can 
contribute to efforts at addressing the opioid crisis in the following ways: 
(i) Identify factors related to opioid dependency, (ii) Inform consumers of 
opioid medication about relevant risk factors, and (iii) Increase knowledge 
of opioid abuse for more informed decisions by prescribers. 

%%%%%%%%%%%%%%%%%%%%%%%%%%%%%%%%%%%%%%%%%%%%%%%%%%%%%%%%%%%%%%%%%%%%%%%%%%%%%%%%
\subsection{Machine Learning Classsification} 

Machine learning describes a set of procedures and automated processes for 
extracting knowledge from data. The two main branches of machine learning are'
supervised learning and unsupervised learning. When a problem has a specific 
target variable or outcome, we are dealing with a supervised learning problem. 
If the data that we are working with has no specific outcome, then we are
dealing with an unsupervised learning task that involves clustering or 
dimension reduction. The goals of this project are primarily direct at the 
outcome of opioid addiction, and will focus on supervised learning problems. 
Supervised learning is used to predict a certain outcome from a given input,
when examples of input/output pairs are available \cite{muller17}. A machine
learning model is constructed from the training set of input-output pairs, 
with the goal of predicting new test data that the model has not seen before. 
The two major approaches to supervised learning problems are regression and
classification. When the target variable to be predicted is continuous, or
there is continuity between the outcome (e.g., home values, or income), a 
regression model is used to test the set of features that predict the target 
variable of interest. If the target to be predicted is a class label, set of 
categorical or binary outcome (e.g., `spam` or `ham`, `benign` or `malignant`), 
then classification is used to predict which class or category label that 
new instances will be assigned to. \cite{raschka17}

%%%%%%%%%%%%%%%%%%%%%%%%%%%%%%%%%%%%%%%%%%%%%%%%%%%%%%%%%%%%%%%%%%%%%%%%%%%%%%%%
\subsection{Project Goals} 
This project will use a set of demographic characteristics, health, and 
psychological characteristics (e.g., adult depression) to classify individuals 
according to opioid addiction. 


%%%%%%%%%%%%%%%%%%%%%%%%%%%%%%%%%%%%%%%%%%%%%%%%%%%%%%%%%%%%%%%%%%%%%%%%%%%%%%%%
\section{Method}

\subsection{National Survey on Drug Use and Health} 

The data used for this project is the 2015 National Survey on Drug Use and 
Health (NSHUH-2015) obtained from the Substance Abuse and Mental Health Data 
Archive \cite{samhsa16}. The data consisted of a sample of 57,146 observations 
with 2,666 features representing individual-level responses to a comprehensive 
survey on substance use (prescription medication and illicit drugs), health, 
depression, employment, and demographic characteristics in the U.S. population. 
Sampling was weighted across states by population size for a representative 
distribution selected from 6,000 area segments varying in size according to 
state. The sample design used five state sample size groups drawing more 
heavily from the eight states with the largest population (e.g., CA, FL, IL, MI, 
NY, OH, PA, TX) which together account for 48 percent of total U.S. population 
aged 12 or older. Given that NSDUH collects highly-sensitive information from 
individuals, all identifying information was collapsed (e.g., age categories)
and state identifiers were removed from the public use file to ensure 
confidentiality. The NSDUH public-use files do not include geographic location, 
or demographic variables related to ethnicity or immigration status. The 
weighted survey screening response rate was 81.94 percent and the weighted
interview response rate was 71.2 percent. The NSDUH-2015 dataset was 
downloaded from the SAMH Data Archive as tab-separated Excel files.


%%%%%%%%%%%%%%%%%%%%%%%%%%%%%%%%%%%%%%%%%%%%%%%%%%%%%%%%%%%%%%%%%%%%%%%%%%%%%%%%


\subsection{Data Cleaning and Preparation}



\cite{mckinney17}






%%%%%%%%%%%%%%%%%%%%%%%%%%%%%%%%%%%%%%%%%%%%%%%%%%%%%%%%%%%%%%%%%%%%%%%%%%%%%%%%
\section{Results}

\subsection{Exploratory Analysis and Descriptives}

\cite{rahm00}


\cite{mckinney17}



%%%%%%%%%%%%%%%%%%%%%%%%%%%%%%%%%%%%%%%%%%%%%%%%%%%%%%%%%%%%%%%%%%%%%%%%%%%%%%%%
\section{Machine Learning Classification Models}

\cite{muller17}


\cite{raschka17}

Machine learning classification 
of opioid abuse can contribute to efforts to address prescription opioid 
addiction, overdoses, in the following ways: 
\begin{enumerate}
\item Identify factors related to opioid dependency
\item Inform consumers of opioid medication as to risk factors 
\item Increase knowledge of opioid abuse for more informed prescriptions. 
\end{enumerate}


\cite{vanderplas17}


%%%%%%%%%%%%%%%%%%%%%%%%%%%%%%%%%%%%%%%%%%%%%%%%%%%%%%%%%%%%%%%%%%%%%%%%%%%%%%%%
\section{Discussion}

\subsection{Classification of Opioid Addiction}








%%%%%%%%%%%%%%%%%%%%%%%%%%%%%%%%%%%%%%%%%%%%%%%%%%%%%%%%%%%%%%%%%%%%%%%%%%%%%%%%


\section{figures}

In Figure \ref{f:fly} we show a fly. Please note that because we use
just columwidth that the size of the figure will change to the
columnwidth of the paper once we change the layout to final. CHnaging
the layout to final should not be done by you. All figures will be
listed at the end.

\begin{figure}[!ht]
  \centering\includegraphics[width=\columnwidth]{images/fly.pdf}
  \caption{Example caption}\label{f:fly}
\end{figure}

When copying the example, please do not check in the images from the
examples into your images directory as you will not need them for your
paper. Instead use images that you like to include. If you do not have
any images, do not dreate the images folder.





%%%%%%%%%%%%%%%%%%%%%%%%%%%%%%%%%%%%%%%%%%%%%%%%%%%%%%%%%%%%%%%%%%%%%%%%%%%%%%%%
\section{Tables}



%Table\ref{t:mytable}. 


\begin{table}[htb]
\centering
\caption{My caption}
\label{t:mytabble}
\begin{tabular}{lll}
1 & 2 & 3 \\
\hline
4 & 5 & 6 \\
7 & 8 & 9
\end{tabular}
\end{table}

\section{Long example}

If you like to see a more elaborate example, please look at
report-long.tex. 

%%%%%%%%%%%%%%%%%%%%%%%%%%%%%%%%%%%%%%%%%%%%%%%%%%%%%%%%%%%%%%%%%%%%%%%%%%%%%%%%
\section{Conclusion}

Conclusions and abstracts must not have any citations.

%%%%%%%%%%%%%%%%%%%%%%%%%%%%%%%%%%%%%%%%%%%%%%%%%%%%%%%%%%%%%%%%%%%%%%%%%%%%%%%%
\begin{acks}

  The author would like to thank Dr. Gregor von Laszewski, 
  the Assistant Instructors, Juliette Zurick, Miao Zheng,
  and others who helped to improve this project and report.

\end{acks}

\bibliographystyle{ACM-Reference-Format}
\bibliography{report} 


%%%%%%%%%%%%%%%%%%%%%%%%%%%%%%%%%%%%%%%%%%%%%%%%%%%%%%%%%%%%%%%%%%%%%%%%%%%%%%%%
\appendix

\section{Code References}
References used in data cleaning and preparation, exploratory analysis, 
visualization, classification model construction and evaluation are documented
here as well as included in comments in the jupyter notebooks 
\cite{muller17,mckinney17,vanderplas17}

Appendix of common issues: 

%\begin{enumerate}

%\item Do not to use the underscore in bibtex labels
%\item Address each of the items in the issues.tex file  and verify you have done them. 
%item Please do this only at the end once you have finished writing the paper. 
%\item Change `TODO` with `DONE`. 

%\end{enumerate}

%\section{Issues}

\DONE{Example of done item: Once you fix an item, change TODO to DONE}

\subsection{Assignment Submission Issues}

    \DONE{Do not make changes to your paper during grading, when your repository should be frozen.}

\subsection{Uncaught Bibliography Errors}

    \DONE{Missing bibliography file generated by JabRef}
    \DONE{Bibtex labels cannot have any spaces, \_ or \& in it}
    \DONE{Citations in text showing as [?]: this means either your report.bib is not up-to-date or there is a spelling error in the label of the item you want to cite, either in report.bib or in report.tex}

\subsection{Formatting}

    \DONE{Incorrect number of keywords or HID and i523 not included in the keywords}
    \DONE{Other formatting issues}

\subsection{Writing Errors}

    \DONE{Errors in title, e.g. capitalization}
    \DONE{Spelling errors}
    \DONE{Are you using {\em a} and {\em the} properly?}
    \DONE{Do not use phrases such as {\em shown in the Figure below}. Instead, use {\em as shown in Figure 3}, when referring to the 3rd figure}
    \DONE{Do not use the word {\em I} instead use {\em we} even if you are the sole author}
    \DONE{Do not use the phrase {\em In this paper/report we show} instead use {\em We show}. It is not important if this is a paper or a report and does not need to be mentioned}
    \DONE{If you want to say {\em and} do not use {\em \&} but use the word {\em and}}
    \DONE{Use a space after . , : }
    \DONE{When using a section command, the section title is not written in all-caps as format does this for you}\begin{verbatim}\section{Introduction} and NOT \section{INTRODUCTION} \end{verbatim}

\subsection{Citation Issues and Plagiarism}

    \DONE{It is your responsibility to make sure no plagiarism occurs. The instructions and resources were given in the class}
    \DONE{Claims made without citations provided}
    \DONE{Need to paraphrase long quotations (whole sentences or longer)}
    \DONE{Need to quote directly cited material}

\subsection{Character Errors}

    \DONE{Erroneous use of quotation marks, i.e. use ``quotes'' , instead of " "}
    \DONE{To emphasize a word, use {\em emphasize} and not ``quote''}
    \DONE{When using the characters \& \# \% \_  put a backslash before them so that they show up correctly}
    \DONE{Pasting and copying from the Web often results in non-ASCII characters to be used in your text, please remove them and replace accordingly. This is the case for quotes, dashes and all the other special characters.}
    \DONE{If you see a figure and not a figure in text you copied from a text that has the fi combined as a single character}

\subsection{Structural Issues}

    \DONE{Acknowledgement section missing}
    \DONE{Incorrect README file}
    \DONE{In case of a class and if you do a multi-author paper, you need to add an appendix describing who did what in the paper}
    \DONE{The paper has less than 2 pages of text, i.e. excluding images, tables and figures}
    \DONE{The paper has more than 6 pages of text, i.e. excluding images, tables and figures}
    \DONE{Do not artificially inflate your paper if you are below the page limit}

\subsection{Details about the Figures and Tables}

    \DONE{Capitalization errors in referring to captions, e.g. Figure 1, Table 2}
    \DONE{Do use {\em label} and {\em ref} to automatically create figure numbers}
    \DONE{Wrong placement of figure caption. They should be on the bottom of the figure}
    \DONE{Wrong placement of table caption. They should be on the top of the table}
    \DONE{Images submitted incorrectly. They should be in native format, e.g. .graffle, .pptx, .png, .jpg}
    \DONE{Do not submit eps images. Instead, convert them to PDF}

    \DONE{The image files must be in a single directory named "images"}
    \DONE{In case there is a powerpoint in the submission, the image must be exported as PDF}
    \DONE{Make the figures large enough so we can read the details. If needed make the figure over two columns}
    \DONE{Do not worry about the figure placement if they are at a different location than you think. Figures are allowed to float. For this class, you should place all figures at the end of the report.}
    \DONE{In case you copied a figure from another paper you need to ask for copyright permission. In case of a class paper, you must include a reference to the original in the caption}
    \DONE{Remove any figure that is not referred to explicitly in the text (As shown in Figure ..)}
    \DONE{Do not use textwidth as a parameter for includegraphics}
    \DONE{Figures should be reasonably sized and often you just need to
    \DONE{includegraphics[width=\columnwidth]{images/myimage.pdf}}
=======
    \TODO{Do not make changes to your paper during grading, when your repository should be frozen.}

\subsection{Uncaught Bibliography Errors}

    \TODO{Missing bibliography file generated by JabRef}
    \TODO{Bibtex labels cannot have any spaces, \_ or \& in it}
    \TODO{Citations in text showing as [?]: this means either your report.bib is not up-to-date or there is a spelling error in the label of the item you want to cite, either in report.bib or in report.tex}

\subsection{Formatting}

    \TODO{Incorrect number of keywords or HID and i523 not included in the keywords}
    \TODO{Other formatting issues}

\subsection{Writing Errors}

    \TODO{Errors in title, e.g. capitalization}
    \TODO{Spelling errors}
    \TODO{Are you using {\em a} and {\em the} properly?}
    \TODO{Do not use phrases such as {\em shown in the Figure below}. Instead, use {\em as shown in Figure 3}, when referring to the 3rd figure}
    \TODO{Do not use the word {\em I} instead use {\em we} even if you are the sole author}
    \TODO{Do not use the phrase {\em In this paper/report we show} instead use {\em We show}. It is not important if this is a paper or a report and does not need to be mentioned}
    \TODO{If you want to say {\em and} do not use {\em \&} but use the word {\em and}}
    \TODO{Use a space after . , : }
    \TODO{When using a section command, the section title is not written in all-caps as format does this for you}\begin{verbatim}\section{Introduction} and NOT \section{INTRODUCTION} \end{verbatim}

\subsection{Citation Issues and Plagiarism}

    \TODO{It is your responsibility to make sure no plagiarism occurs. The instructions and resources were given in the class}
    \TODO{Claims made without citations provided}
    \TODO{Need to paraphrase long quotations (whole sentences or longer)}
    \TODO{Need to quote directly cited material}

\subsection{Character Errors}

    \TODO{Erroneous use of quotation marks, i.e. use ``quotes'' , instead of " "}
    \TODO{To emphasize a word, use {\em emphasize} and not ``quote''}
    \TODO{When using the characters \& \# \% \_  put a backslash before them so that they show up correctly}
    \TODO{Pasting and copying from the Web often results in non-ASCII characters to be used in your text, please remove them and replace accordingly. This is the case for quotes, dashes and all the other special characters.}
    \TODO{If you see a figure and not a figure in text you copied from a text that has the fi combined as a single character}

\subsection{Structural Issues}

    \TODO{Acknowledgement section missing}
    \TODO{Incorrect README file}
    \TODO{In case of a class and if you do a multi-author paper, you need to add an appendix describing who did what in the paper}
    \TODO{The paper has less than 2 pages of text, i.e. excluding images, tables and figures}
    \TODO{The paper has more than 6 pages of text, i.e. excluding images, tables and figures}
    \TODO{Do not artificially inflate your paper if you are below the page limit}

\subsection{Details about the Figures and Tables}

    \TODO{Capitalization errors in referring to captions, e.g. Figure 1, Table 2}
    \TODO{Do use {\em label} and {\em ref} to automatically create figure numbers}
    \TODO{Wrong placement of figure caption. They should be on the bottom of the figure}
    \TODO{Wrong placement of table caption. They should be on the top of the table}
    \TODO{Images submitted incorrectly. They should be in native format, e.g. .graffle, .pptx, .png, .jpg}
    \TODO{Do not submit eps images. Instead, convert them to PDF}

    \TODO{The image files must be in a single directory named "images"}
    \TODO{In case there is a powerpoint in the submission, the image must be exported as PDF}
    \TODO{Make the figures large enough so we can read the details. If needed make the figure over two columns}
    \TODO{Do not worry about the figure placement if they are at a different location than you think. Figures are allowed to float. For this class, you should place all figures at the end of the report.}
    \TODO{In case you copied a figure from another paper you need to ask for copyright permission. In case of a class paper, you must include a reference to the original in the caption}
    \TODO{Remove any figure that is not referred to explicitly in the text (As shown in Figure ..)}
    \TODO{Do not use textwidth as a parameter for includegraphics}
    \TODO{Figures should be reasonably sized and often you just need to
    \TODO{Includegraphics[width=\columnwidth]{images/myimage.pdf}}

re


\end{document}
